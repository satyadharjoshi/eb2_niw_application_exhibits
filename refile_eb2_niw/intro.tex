
The undersigned represents Satyadhar Joshi, an experienced researcher currently serving as Assistant Vice President (AVP) in the Global Risk \& Analytics division at Bank of America, Jersey City, New Jersey. This petition seeks classification under the Employment-Based Second Preference (EB2) category, accompanied by a request for a National Interest Waiver (NIW) of the job offer and labor certification requirement.

Mr. Joshi’s research lies at the intersection of Agentic Generative AI, Financial Risk, Big Data, and Policy Systems, addressing challenges of national importance across finance, defense, healthcare, and workforce transformation. His contributions at a systemically important financial institution strengthen the resilience, transparency, and security of the U.S. financial system. In parallel, his independent research promotes the responsible design and governance of Agentic AI frameworks that align with U.S. priorities in innovation, economic security, and workforce competitiveness.

Looking ahead, Mr. Joshi intends to transition from the private sector into the public research domain, focusing on the creation of / or becoming partner at a dedicated Agentic AI Policy and Workforce Research Center or collaboration with a leading think tank. His planned research program centers on AI governance, systemic risk in autonomous systems, ethical deployment, and national workforce realignment for the AI-driven economy. He aims to develop evidence-based models and data-driven frameworks to support retraining and upskilling initiatives that prepare U.S. workers—including veterans and mid-career professionals—for emerging roles in the AI and automation economy.

By advancing these objectives through open, collaborative, and transparent research, Mr. Joshi’s work will help ensure that AI systems deployed in critical sectors remain safe, interpretable, and socially beneficial. Granting the National Interest Waiver will enable him to pursue this independent, mission-driven research outside commercial constraints, thereby amplifying its national value. His transition from corporate analytics to public-interest AI and workforce policy research will yield lasting contributions to U.S. technological competitiveness, national security, and inclusive economic growth.

This petition demonstrates that Mr. Joshi satisfies all regulatory criteria for the EB2 classification and meets the three-prong test established in \textit{Matter of Dhanasar}:

\begin{enumerate}
	\item His proposed endeavor has both \textbf{substantial merit} and \textbf{national importance}.
	\item He is \textbf{well-positioned} to advance the proposed endeavor.
	\item On balance, it would be \textbf{beneficial to the United States} to waive the job offer and labor certification requirements.
\end{enumerate}

\chapter{Legal Criteria and Eligibility}

Under USCIS guidelines, the EB2 National Interest Waiver is available to professionals who hold an advanced degree or its equivalent, or who possess exceptional ability in the sciences, arts, or business, and whose work substantially benefits the national interest of the United States.

\subsection{Advanced Degree and Equivalent Qualifications (8 CFR § 204.5(k)(2))}
Mr. Joshi possesses the required advanced academic credentials. Refer to Exhibit \ref{chap:education}:
\begin{itemize}
	\item \textbf{Master of Science in Information Systems} from Touro College, New York  
	\item \textbf{Master of Business Administration (MBA)} from Bar Ilan University, Israel, with a specialization in cross-cultural business policies and global management  
\end{itemize}


\section{International Government and Quasi-Government Recognition of Research}

The following section documents the recognition and indexing of Mr. Joshi's research by governmental and quasi-governmental bodies worldwide, demonstrating the international reach and authoritative validation of his work in artificial intelligence, financial risk management, and workforce development.

\subsection{United States Government Recognition of Mr Joshi's Work}

\subsubsection{U.S. Department of Energy - Science.gov}
\begin{itemize}
	\item \textbf{Platform}: Science.gov, the official portal for U.S. government science information
	\item \textbf{Managing Agency}: Office of Science and Technical Information (OSTI) under U.S. Department of Energy
	\item \textbf{Significance}: Indexing indicates research relevance to national priorities and federal agency interests (NSF, DOE, NIH, NASA, etc.)
	\item \textbf{Research Indexed }: 
	
		Pathak, R., Joshi, S. (2009). Multi-scale Modeling and Analysis of Nano-RFID Systems on HPC Setup. In: Ranka, S., et al. Contemporary Computing. IC3 2009. Communications in Computer and Information Science, vol 40. Springer, Berlin, Heidelberg. 
	Pathak, R., Joshi, S. (2009). Secure Multi-party Computation Protocol for Defense Applications in Military Operations Using Virtual Cryptography. In: Ranka, S., et al. Contemporary Computing. IC3 2009. Communications in Computer and Information Science, vol 40. Springer, Berlin, Heidelberg.
	Pathak, R., Joshi, S. (2009). Secured Communication for Business Process Outsourcing Using Optimized Arithmetic Cryptography Protocol Based on Virtual Parties. In: Ranka, S., et al. Contemporary Computing. IC3 2009. Communications in Computer and Information Science, vol 40. Springer, Berlin, Heidelberg.
	

	
	\item \textbf{Verification}: Accessible through official science.gov domain searches, Springer Link Website and Exhibits
	~\ref{chap:exhibit_sgov} 
\end{itemize}

\subsubsection{U.S. Federal Reserve Board}
\begin{itemize}
	\item \textbf{Research Cited }: Joshi, Satyadhar, “Generative AI in Investment and Portfolio Management: Comprehensive Review of Current Applications and Future Directions,” Technical Report, preprints.org 2025.
	\item \textbf{Citation of Research}: Research cited in Finance and Economics Discussion Series paper "Generative AI at the Crossroads: Light Bulb, Dynamo, or Microscope?" (Baily et al., June 27, 2025)
	\item \textbf{Significance}: Direct relevance to critical discussions on AI's impact on national economy
	\item \textbf{Impact}: Informs Federal Reserve policy research and economic analysis
	\item \textbf{Verification}: Accessible through official .gov domain searches and Exhibit ~\ref{chap:federal_reserve}

\end{itemize}

\subsubsection{U.S. Bureau of Labor Statistics (BLS)}
\begin{itemize}
	\item \textbf{Research Cited }: Satyadhar Joshi, “Generative AI: Mitigating Workforce and Economic Disruptions While Strategizing Policy Responses for Governments and Companies,” Journal of Advanced Research in Science, Communication and Technology (IJARSCT), vol. 5, no. 1, p. 480-486, Feb. 2025, doi: 10.48175/IJARSCT-23260.
	\item \textbf{Recognition}: Research utilized in BLS publication on workforce and AI policy (DOI: 10.21916/mlr.2022.21)
	\item \textbf{Metadata Evidence}: Backend YAML data explicitly lists Mr. Joshi as contributing author
	\item \textbf{Significance}: Direct impact on U.S. government workforce research and policy considerations
	\item \textbf{Verification}: Accessible through official .gov domain searches and Exhibit ~\ref{chap:bls}
\end{itemize}



\subsubsection{U.S. Department of Education - ERIC (Institute of Education Sciences)}

\begin{itemize}
	\item \textbf{Platform}: ERIC (Education Resources Information Center)
	\item \textbf{Managing Agency}: Institute of Education Sciences, U.S. Department of Education
	\item \textbf{ERIC Number}: ED676035
	\item \textbf{Research Indexed}: "Enhancing U.S. K-12 Competitiveness for the Agentic Generative AI Era: A Structured Framework for Educators and Policy Makers" (October 2025)
	\item \textbf{Significance}: Official recognition by the U.S. Department of Education's primary education research database, demonstrating direct relevance to national education policy and K-12 reform initiatives
	\item \textbf{Verification}: Accessible through official ERIC database at \url{https://eric.ed.gov/} Exhibit ~\ref{chap:eric}
\end{itemize}

\subsection{International Government Recognition}



\subsubsection{France - Ministère de l'Enseignement supérieur, de la Recherche et de l'Espace}
\begin{itemize}
	\item \textbf{Platform}: scanR
	\item \textbf{Operator}: French Ministry of Higher Education, Research and Innovation
	\item \textbf{Verification}: \url{https://scanr.enseignementsup-recherche.gouv.fr/search/publications?q=satyadhar+joshi}
	\item \textbf{Significance}: Official discovery portal for French research and innovation, operated directly by the national government.
	\item \textbf{Research Indexed}: 16 publications by Mr Satyadhar Joshi
	\item \textbf{Notable Works}: Includes foundational reviews on "Model Risk Management in the Era of Generative AI," "DeepSeek: Performance and Architecture," and "Mixture of Experts Models in Business and Finance."
	\item \textbf{Impact}: Integration into a national-level research infrastructure demonstrates recognition and dissemination of research to French academic, governmental, and innovation ecosystems.
	\item 
	Refer to Exhibits	~\ref{chap:scanr} 
	
	
\end{itemize}


\subsubsection{Germany - Munich Personal RePEc Archive (MPRA)}
\begin{itemize}
	\item \textbf{Platform}: Munich Personal RePEc Archive (MPRA)
	\item \textbf{Operator}: University of Munich Library (Ludwig-Maximilians-Universität), a leading German research university
	\item \textbf{Verification}: \url{https://mpra.ub.uni-muenchen.de/id/eprint/125221/}
	\item \textbf{Research Hosted}: \textit{Joshi, Satyadhar (2025): Model Risk Management in the Era of Generative AI: Challenges, Opportunities, and Future Directions. Published in: International Journal of Scientific and Research Publications , Vol. 5, No. 15 (20 May 2025): pp. 299-309.} 
	\item \textbf{Significance}: Integration into a major European university's research archive, ensuring broad dissemination and permanent accessibility to the global economics and finance research community.
	\item \textbf{Impact}: The paper provides a critical synthesis of regulatory frameworks and quantitative risk metrics (including probabilistic frameworks and adversarial risk calculations) essential for the stability of financial institutions adopting Generative AI, a subject of key importance to U.S. and global financial security.
	
		\item 
	Refer to Exhibits	~\ref{chap:mpra} 
\end{itemize}


\subsubsection{China - SciEngine (China Science Publishing \& Media Ltd.)}
\begin{itemize}
	\item \textbf{Platform}: SciEngine, operated by China Science Publishing \& Media Ltd. (Science Press)
	\item \textbf{Affiliation}: State-owned academic publisher affiliated with Chinese Academy of Sciences
	\item \textbf{Citation}: Work cited in article available at: \url{https://www.sciengine.com/BNSFC/doi/10.3724/BNSFC-2025.04.20.0001}
	\item \textbf{Significance}: International recognition through government-affiliated publishing outlet
	\item \textbf{Impact}: Demonstrated reliance by scholars in government-affiliated research institutions
	\item \textbf{Research cited}: Joshi, Satyadhar. "Review of gen ai models for financial risk management." International Journal of Scientific Research in Computer Science, Engineering and Information Technology 11, no. 1 (2025): 709-723.  https://doi.org/10.32628/CSEIT2511114 
\end{itemize}

\subsubsection{United Kingdom - CORE.ac.uk}
\begin{itemize}
	\item \textbf{Platform}: CORE (UK), United Kingdom's open-access research aggregator
	\item \textbf{Operator}: Knowledge Media Institute at The Open University
	\item \textbf{Usage}: Widely used by UK Research and Innovation (UKRI) and Research England
	\item \textbf{Research Hosted}: Several https://core.ac.uk/search/?q=satyadhar\%20joshi
	
	
	\item \textbf{Significance}: Broad dissemination to UK academic, policy, and industry stakeholders
\item
	Refer to Exhibits	~\ref{chap:core_uk} 

\end{itemize}

\subsubsection{Ukraine - Open Ukrainian Citation Index (OUCI)}
\begin{itemize}
	\item \textbf{Platform}: Open Ukrainian Citation Index (OUCI)
	\item \textbf{Operator}: State Scientific and Technical Library of Ukraine (DNTB) under Ukrainian government authority
	\item \textbf{Research Indexed}: 
	
	
	Generative AI in Investment and Portfolio Management: Comprehensive Review of Current Applications and Future Directions https://ouci.dntb.gov.ua/en/works/9JQQ2qom/
	
	Artificial Intelligence in Conflict Resolution: A Comprehensive Review of Techniques and Applications https://ouci.dntb.gov.ua/en/works/lRrrVD0E/
	
	Review of Artificial General Intelligence (AGI): Implications for the U.S. Workforce and Economic Stability https://ouci.dntb.gov.ua/en/works/lDdd6k6z/
	
	Retraining US Workforce in the Age of Agentic Gen AI: Role of Prompt Engineering and Up-Skilling Initiatives https://ouci.dntb.gov.ua/en/works/45rNeoZE/
	
	\item \textbf{Significance}: Official Ukrainian platform increasing accessibility and visibility
	\item \textbf{Relevance}: Contribution to global discussions on AI and conflict resolution
	\item
	Refer to Exhibits	~\ref{chap:ouci} 
	
\end{itemize}


\subsubsection{Europe PMC}
\begin{itemize}
	\item \textbf{Platform}: Europe PMC
	\item \textbf{Operator}: EMBL-EBI with support from the Europe PMC Funders' Group, in collaboration with the National Library of Medicine (NLM)
	\item \textbf{Over 63+ Preprints Research Indexed}: 
	
	GenAI Agents for Early Disease Diagnosis: A Review of Architectures, Applications, and Policy Directions \url{https://europepmc.org/article/PPR/PPR1099612}
		
	\item \textbf{Significance}: Major European repository for life sciences literature with international collaboration
	
		\item \textbf{Verification}: Accessible through official ERIC database at \url{https://europepmc.org/search?query=satyadhar%20joshi} Exhibits
		~\ref{chap:eu_pmc} 
	
\end{itemize}


\subsection{Quasi-Government and Institutional Recognition}

\subsubsection{Academic Integration - International Institutions}
\begin{itemize}
	\item \textbf{Zuyd University of Applied Sciences (Netherlands)}: Research integrated into academic curricula and research guides
	\item \textbf{Harrisburg University Digital Commons (USA)}: Inclusion in institutional research repositories. 	Refer to Exhibits	~\ref{chap:harrisburg} 
	\item \textbf{Significance}: Adoption by international educational institutions demonstrates academic validation
\end{itemize}

\subsubsection{Government-Affiliated Indexing Services}
\begin{itemize}
	\item \textbf{Index Copernicus}: European journal indexing system evaluating publication quality
	\item \textbf{Econ Papers}: Economics research database recognition
	\item \textbf{Significance}: International scholarly validation through government-affiliated indexing services
\end{itemize}

\subsection{Summary of Government Recognition Impact}

\begin{table}[h]
	\centering
	\caption{Government and Quasi-Government Recognition Summary}
	\begin{tabular}{|p{3cm}|p{4cm}|p{4cm}|}
		\hline
		\textbf{Government Entity} & \textbf{Type of Recognition} & \textbf{National Importance Demonstrated} \\
		\hline
		U.S. Department of Energy & Science.gov indexing & Relevance to federal scientific priorities \\
		\hline
		U.S. Federal Reserve & Research citation & Impact on economic policy research \\
		\hline
		U.S. Bureau of Labor Statistics & Research utilization & Influence on workforce policy development \\
		\hline
		China Science Publishing & SciEngine citation & International scholarly recognition \\
		\hline
		UK Research Institutions & CORE.ac.uk hosting & Dissemination to UK policy and academic circles \\
		\hline
		Ukrainian Government & OUCI indexing & Global relevance in conflict resolution AI \\
		\hline
		U.S. Department of Education & ERIC database indexing & Impact on national education policy and K-12 AI integration \\
		\hline

	\end{tabular}
\end{table}

\subsection{Conclusion on Government Recognition}

The extensive recognition of Mr. Joshi's research by multiple governmental and quasi-governmental bodies across the United States, China, United Kingdom, Ukraine, and European institutions demonstrates:

\begin{itemize}
	\item \textbf{International Reach}: Global acknowledgment of research quality and relevance
	\item \textbf{Policy Impact}: Direct influence on government research and policy considerations
	\item \textbf{Substantial Merit}: Validation by authoritative governmental bodies
	\item \textbf{National Importance}: Alignment with critical national priorities in multiple countries
	\item \textbf{Research Quality}: Meeting rigorous standards required for government indexing and citation
\end{itemize}

This multi-national government recognition provides compelling evidence that Mr. Joshi's work possesses both substantial merit and national importance, satisfying the highest standards of the Dhanasar framework for EB-2 NIW classification.

\section{Demonstrated Influence and Dissemination of Research at US Universities}

The work of Satyadhar Joshi has achieved significant recognition, as evidenced by its dissemination through US .EDU academic repositories and citation across multiple educational institutions. This broad uptake across the academic community underscores the importance and utility of his contributions.

\subsection*{Indexing in Major Academic Repositories}

Mr. Joshi's research is formally indexed and permanently archived in multiple university repositories, ensuring its accessibility to the global research community.

\begin{itemize}
	\item His recent review paper on artificial intelligence is published and indexed in the \textbf{Harrisburg University Digital Commons}\footnote{Joshi, S. (2025). \textit{The Role of AI in Enhancing Teamwork, Resilience and Decision-Making}. Harrisburg University Digital Commons. https://digitalcommons.harrisburgu.edu/}, demonstrating immediate impact through early citations. 	Refer to Exhibits	~\ref{chap:harrisburg} 
	
	\item His collaborative work on plasmonic solar cells is permanently archived in the \textbf{Michigan Technological University Digital Commons}\footnote{Vora, A., Joshi, S., et al. (2018). \textit{Optimal design of thin-film plasmonic solar cells}. Michigan Tech Digital Commons. https://digitalcommons.mtu.edu/michigantech-p/2109}, representing significant engineering research in renewable energy.
	
	\item His early research on secure computation protocols is indexed in the \textbf{Smithsonian Astrophysical Observatory database}\footnote{Pathak, R. \& Joshi, S. (2009). \textit{Secure Multi-party Computation Protocol for Defense Applications}. NASA Astrophysics Data System. https://ui.adsabs.harvard.edu/} hosted by Harvard University, indicating cross-disciplinary relevance and high academic caliber.
\end{itemize}

\subsection*{Citation and Utilization Across Educational Institutions}

The practical value and scholarly influence of Mr. Joshi's work are demonstrated by its utilization across multiple educational contexts.

\begin{itemize}
	\item His optimization research has been cited and utilized as a technical reference in senior design projects at \textbf{California State University, Sacramento}\footnote{Team LOKSYS (2015). \textit{Senior Design Project Report}. California State University, Sacramento. https://www.csus.edu/indiv/t/tatror/senior\_design/SD\%20F14-S15/Team\_2\_LOKSYS\_F14\_to\_S15.pdf}, showing direct pedagogical impact on engineering education.
	
	\item Furthermore, his work has been referenced in publications from the \textbf{Massachusetts Institute of Technology (MIT)}\footnote{MIT Center for Transportation \& Logistics (2018). \textit{Research Publication}. Massachusetts Institute of Technology. https://sheffi.mit.edu/sites/sheffi.mit.edu/files/2018-07/11\_09574090410700194.pdf}, demonstrating influence at premier research institutions.
\end{itemize}


\subsection{Conference Publication Indexing in Stanford.edu}

A notable early contribution in the domain of secure computation and data privacy was presented by Satyadhar Joshi and Rohit Pathak in their paper titled \textit{``Secure Multi-Party Computation Protocol for Statistical Computation on Encrypted Data''}. The work was featured in the \textbf{Proceedings of the 2009 International Conference on Software Technology and Engineering (ICSTE 2009)}, held in Chennai, India, from 24–26 July 2009. 

This paper proposed a protocol enabling statistical computations on encrypted datasets using multi-party computation principles, thereby enhancing data confidentiality in distributed environments. The publication is accessible through the Stanford University Libraries repository at:

\begin{quote}
	\url{https://searchworks.stanford.edu/view/12926240}
\end{quote}

The full citation is as follows:

\begin{quote}
	R. Pathak and S. Joshi, ``Secure Multi-Party Computation Protocol for Statistical Computation on Encrypted Data,'' in \textit{Proceedings of the 2009 International Conference on Software Technology and Engineering}, Chennai, India, 24–26 July 2009, pp. xvii–394.
\end{quote}



subsection*{Indexing in Major Academic Repositories}

Mr. Joshi's research is formally indexed and permanently archived in multiple university repositories, ensuring long-term accessibility to the global research community.

\begin{table*}[h!]
	\caption{U.S. University Repositories and Academic Indexing of Publications by Satyadhar Joshi}
	\label{tab:university_indexing}
	\centering
	\begin{tabular}{|p{3.5cm}|p{4cm}|p{5.5cm}|p{3cm}|}
		
		
		
		
		
		
		\hline
		\textbf{University / Repository} & \textbf{Title of Work} & \textbf{Focus Area} & \textbf{Access / Link} \\
		
		
		\hline
		Harrisburg University Digital Commons & \textit{Advancing U.S. Competitiveness in Agentic Gen AI: A Strategic Framework for Interoperability and Governance} (2025) & Agentic AI governance, U.S. competitiveness, and strategic frameworks for AI leadership & \url{https://digitalcommons.harrisburgu.edu/other-works/14/} \\
		\hline
		
		\hline
		Harrisburg University Digital Commons & \textit{Leadership in the age of AI: Review of quantitative models and visualization for managerial decision-making} (2025) & AI-driven leadership models, quantitative decision-making, and organizational transformation & \url{https://digitalcommons.harrisburgu.edu/other-works/13/} \\
		\hline
		
		
\hline
Harrisburg University Digital Commons & \textit{The Role of AI in Enhancing Teamwork, Resilience and Decision-Making: Review of Recent Developments} (2025) & AI-enhanced organizational behavior, team resilience, and human-AI collaborative decision-making & \url{https://digitalcommons.harrisburgu.edu/other-works/10/} \\
\hline


		\hline
		Michigan Technological University Digital Commons & \textit{Optimal Design of Thin-Film Plasmonic Solar Cells} (2018) & Renewable energy optimization and nanophotonic engineering research & \url{https://digitalcommons.mtu.edu/michigantech-p/2109} \\
		\hline
		Smithsonian / Harvard University (NASA ADS Database) & \textit{Secure Multi-party Computation Protocol for Defense Applications} (2009) & Secure computation and encrypted data analysis for defense systems & \url{https://ui.adsabs.harvard.edu/} \\
		\hline
		Stanford University Libraries (SearchWorks) & \textit{Secure Multi-Party Computation Protocol for Statistical Computation on Encrypted Data} (2009) & Privacy-preserving computation and data security in distributed environments & \url{https://searchworks.stanford.edu/view/12926240} \\
		\hline
	\end{tabular}
\end{table*}

\subsection*{Citation and Utilization Across Educational Institutions}

The scholarly and practical impact of Mr. Joshi’s work extends to multiple academic and instructional contexts.

\begin{table*}[h!]
	\caption{Citation and Utilization of Research Across Educational Institutions}
	\label{tab:citation_utilization}
	\centering
	\begin{tabular}{|p{3.5cm}|p{4cm}|p{5.5cm}|p{3cm}|}
		\hline
		\textbf{Institution} & \textbf{Use Case / Publication} & \textbf{Context of Utilization} & \textbf{Access / Link} \\
		\hline
		California State University, Sacramento & Reference of Rohit Pathak, Satyadhar Joshi, “Internet, 2009 First Asian Himalays International Conference on,”(Recent Trends in
		RFID and a Java based Software Framework for its Integration in Mobile Phones) in  Senior Design Project Report (2015) & Applied optimization research used in undergraduate engineering design coursework & \url{https://www.csus.edu/indiv/t/tatror/senior_design/SD%20F14-S15/Team_2_LOKSYS_F14_to_S15.pdf} \\
		\hline
		Massachusetts Institute of Technology (MIT) & Reference of R. Pathak and S. Joshi, "Recent trends in RFID and a java based software framework for its integration in mobile phones," 2009 First Asian Himalayas International Conference on Internet, Kathmundu, Nepal, 2009, pp. 1-5, doi: 10.1109/AHICI.2009.5340296.  in Center for Transportation and Logistics Report (2018) & 
		
	Applied systems optimization and decision analytics research & \url{https://sheffi.mit.edu/sites/sheffi.mit.edu/files/2018-07/11_09574090410700194.pdf} \\
		\hline
	\end{tabular}
\end{table*}

\subsection*{Conclusion}

The indexing of Satyadhar Joshi's publications across multiple university repositories (.edu domains), combined with their citation in both curriculum development at public universities and research at leading technological institutions, provides compelling evidence of his work's widespread acceptance and significant influence within the academic community. 

Mr. Joshi's research is formally indexed and permanently archived within the digital repositories of leading U.S. universities, ensuring its long-term preservation and accessibility to researchers, policymakers, and industry professionals. This institutional adoption underscores the enduring value and relevance of his work to the American research ecosystem. The following table highlights key publications archived by U.S. academic institutions, with a focus on his recent, nationally critical work in Artificial Intelligence.

