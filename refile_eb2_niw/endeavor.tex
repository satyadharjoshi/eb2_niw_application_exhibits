



\label{chap:proposed_endeavor}

\section{Overview}
The applicant proposes a multi-year initiative focused on financial system resilience through advanced AI, open-source research, and workforce development. The following sections detail the activities, timelines, and projected deliverables.

\section{Initiative 1: Advancing Agentic Gen AI in Financial and Economic Risk}
\begin{itemize}
	\item Develop generative AI models for stress testing and monitoring U.S. financial institutions to enhance economic stability.
	\item Implement real-time risk monitoring using big data platforms such as Hadoop and Spark for financial decision support.
	\item Enhance Gen AI implementation, oversight, and regulation in the U.S. healthcare domain to improve safety, compliance, and operational efficiency.
\end{itemize}

\section{Initiative 2: Open-Source Research and Knowledge Dissemination on Agentic Gen AI}
\begin{itemize}
	\item Publish peer-reviewed research on AI applications in financial risk management, healthcare, and cybersecurity for U.S. stakeholders.
	\item Release open-source tools for regulatory compliance, risk modeling, and operational transparency across sectors.
	\item Organize workshops and webinars for financial sector professionals to share best practices and advance AI adoption responsibly.
\end{itemize}

\section{Initiative 3: Workforce Development and Training, Thought Leadership of Gen AI Application in Health, Education and Cyberspace}
\begin{itemize}
	\item Create training programs for U.S. veterans and other professionals to enter AI-driven financial and technology careers.
	\item Develop online courses (e.g., Gen AI Python for Health, Cyberspace, LLMs for compliance) to build AI literacy and skills in health, education, and cybersecurity domains.
	\item Launch a non-profit initiative (CRAF) focused on financial AI policy, ethical AI adoption, and workforce upskilling in the United States.
\end{itemize}




\section{Five-Year Timeline and Milestones}
\begin{longtable}{|p{0.18\textwidth}|p{0.30\textwidth}|p{0.46\textwidth}|}
	\hline
	\textbf{Year} & \textbf{Key Activities} & \textbf{Projected Outcomes} \\
	\hline
	2026 & Launch CRAF non-profit initiative, implement AI risk tools & 5,000+ course registrations, 500+ veterans enrolled \\
	\hline
	2027 & Publish “State of AI” report, deploy risk adoption tools & 20\% drop in risk loss, 10,000+ downloads \\
	\hline
	2028 & Veteran AI fellowship, automate compliance & 15,000+ MOOC enrollments, 30\% cost reduction \\
	\hline
	2029 & AI audit toolkit, policy papers, chatbots & Use by 2+ federal agencies, 20,000+ trainees \\
	\hline
	2030 & Nationwide bootcamp expansion, predictive analytics & 1,000+ veterans trained, 15\% lower defaults \\
	\hline
\end{longtable}

% Add more details as needed; avoid policy arguments or impact statements here.






Mr. Joshi's proposed endeavor focuses on three integrated components that collectively address critical national needs:

\subsection{Advancing Agentic Gen AI System Resilience through AI}
Mr. Joshi will continue developing and implementing advanced AI and machine learning models to enhance risk management capabilities at U.S. financial institutions by migration to his proposed Policy Research Research. His work specifically addresses:
\begin{itemize}
	\item Development of generative AI models for stress testing and scenario analysis
	\item Real-time risk monitoring systems using big data technologies (Hadoop, Spark, Kafka)
	\item Enhanced agentic risk detection and prevention frameworks
	\item Improving model accuracy and reducing operational risks in systemic financial institutions
\end{itemize}

\subsection{Research and Knowledge Dissemination}
Mr. Joshi will continue his scholarly contributions through:
\begin{itemize}
	\item Publication of peer-reviewed research on AI applications in finance
	\item Development of open-source tools for risk modeling and regulatory compliance
	\item Sharing best practices and innovations with the broader financial and regulatory community
\end{itemize}

\subsection{Workforce Development and Education}
Mr. Joshi is committed to enhancing U.S. workforce capabilities through:
\begin{itemize}
	\item Creating educational resources and training programs in financial analytics and AI
	\item Establishing specialized training initiatives for U.S. veterans transitioning to financial technology careers
	\item Developing accessible online courses and materials to upskill American professionals (\exhibit)
\end{itemize}

\section{Substantial Merit and National Importance}

\subsection{Alignment with National Priorities}
Mr. Joshi's work addresses several areas of critical national importance:

\subsubsection{Agentic Gen AI System Stability for Finance, Economics and Health}
The stability of the U.S. financial system is a matter of paramount national importance. Mr. Joshi's work developing advanced risk models directly supports this priority by:
\begin{itemize}
	\item Enhancing the accuracy of risk assessments for major financial institutions
	\item Reducing systemic risks of Agentic Gen AI risk adoption through better predictive modeling
	\item Supporting the mandates of the Financial Stability Oversight Council (FSOC) and U.S. Treasury Department
\end{itemize}

\subsubsection{Technological Innovation and Business Leadership}
The United States has identified leadership in artificial intelligence and big data technologies as a strategic national priority. Mr. Joshi's work contributes to this leadership by:
\begin{itemize}
	\item Applying cutting-edge AI techniques to solve practical financial challenges
	\item Developing innovative approaches to data analysis and model validation
	\item Enhancing the competitiveness of U.S. leadership institutions in global markets
	\item Supporting initiatives outlined in the White House's Executive Order on AI (Oct. 2023)
\end{itemize}

\subsubsection{Workforce Development}
Developing a skilled workforce capable of implementing advanced technologies is essential for national economic competitiveness. Mr. Joshi's educational initiatives address this need by:
\begin{itemize}
	\item Providing specialized training in high-demand technical skills
	\item Creating pathways for veterans to transition to civilian careers in technology
	\item Addressing skills gaps in the financial technology sector
	\item Supporting Department of Labor workforce development goals
\end{itemize}

\subsection{Evidence of Impact and Recognition}

\subsubsection{Professional Impact}
Mr. Joshi's work has demonstrated tangible benefits to U.S. financial institutions:
\begin{itemize}
	\item Developed quantitative models managing hundreds of billions of dollars in assets at Bank of America
	\item Implemented automation processes that reduced errors by 30-50\% at Wells Fargo
	\item Created risk assessment frameworks that enabled preemptive actions during volatile market conditions
	\item Received strong endorsements from industry leaders and supervisors 
\end{itemize}




\section{Future Plans and Projected Impact}

Mr. Joshi has developed detailed plans for advancing his proposed endeavor over the next five years, with specific metrics for measuring impact.

\subsection{Research and Development Goals}
\begin{longtable}{|p{0.3\textwidth}|p{0.65\textwidth}|}
	\hline
	\textbf{Objective} & \textbf{Metrics and Impact} \\
	\hline
	\textbf{Peer-Reviewed Publications} & 
	\begin{itemize}
		\item Publish 3-4 papers annually in high-quality journals
		\item Focus areas: Generative AI for finance, real-time risk monitoring, adversarial robustness in financial models
		\item Target 20-30 paper reviews annually for peer journals
	\end{itemize} \\
	\hline
	\textbf{Public Policy Impact} & 
	\begin{itemize}
		\item Maintain 10,000-20,000 monthly downloads of policy materials
		\item Expand repository to include regulatory sandbox frameworks, AI fairness toolkits, stress testing methodologies
		\item Partner with research institutions on white papers
	\end{itemize} \\
	\hline
\end{longtable}

\subsection{Workforce Development Initiatives}
\begin{longtable}{|p{0.3\textwidth}|p{0.65\textwidth}|}
	\hline
	\textbf{Program} & \textbf{Projected Growth} \\
	\hline
	\textbf{Veterans in Financial AI} & 
	\begin{itemize}
		\item Scale from current 1,000 to 10,000+ learners by 2030
		\item Launch 3 new certification tracks: GenAI for AML compliance, Agentic AI for efficiency, Python for Agentic AI Edge AI
		\item Secure DOL/VA funding for national expansion
	\end{itemize} \\
	\hline
	\textbf{Open Courseware} & 
	\begin{itemize}
		\item Grow registrations significantly
		\item Develop integrated curricula: Python for quant finance (2026), LLM prompt engineering (2027), synthetic data generation (2028)
	\end{itemize} \\
	\hline
\end{longtable}

\subsection{Five-Year Impact Projection}
\begin{longtable}{|p{0.12\textwidth}|p{0.43\textwidth}|p{0.38\textwidth}|}
	\hline
	\textbf{Year} & \textbf{Key Initiatives} & \textbf{Projected Outcomes} \\
	\hline
	\textbf{2026} & 
	\begin{itemize}
		\item Launch Center for Responsible AI in Finance (CRAF)
		\item Publish 4 peer-reviewed papers
		\item Release AI for Financial Risk Management course
		\item Implement AI-based credit risk tools at Bank of America
	\end{itemize} & 
	\begin{itemize}
		\item 5,000+ course registrations
		\item 15\% improvement in model accuracy
		\item 500+ veterans enrolled
	\end{itemize} \\
	\hline
	
	\textbf{2027} & 
	\begin{itemize}
		\item Publish State of AI in U.S. Finance report
		\item Deploy fraud detection models
		\item Develop AI Agents in Banking workshops
		\item Launch open-source tools with regulators
	\end{itemize} & 
	\begin{itemize}
		\item 20\% reduction in fraud losses
		\item 10,000+ paper downloads
		\item 3 community bank partnerships
	\end{itemize} \\
	\hline
	
	\textbf{2028} & 
	\begin{itemize}
		\item Establish veteran fellowship for AI finance
		\item Launch AGI credit risk MOOC
		\item Automate compliance monitoring
		\item Publish 2 papers on AI adoption in Risk
	\end{itemize} & 
	\begin{itemize}
		\item 100+ veteran fellows placed
		\item 30\% reduction in compliance costs
		\item 15,000+ MOOC enrollments
	\end{itemize} \\
	\hline
	
	\textbf{2029} & 
	\begin{itemize}
		\item Release AI audit toolkit for regulators
		\item Publish ethics/compliance training module
		\item Deploy customer-facing AI chatbots
		\item Submit policy paper on AI fairness in lending
	\end{itemize} & 
	\begin{itemize}
		\item Toolkit used by 2+ federal agencies
		\item 25\% boost in customer satisfaction
		\item 20,000+ training completions
	\end{itemize} \\
	\hline
	
	\textbf{2030} & 
	\begin{itemize}
		\item Expand veteran bootcamp nationally
		\item Deploy predictive loan default analytics
		\item Establish 2–5 regional training hubs
		\item Publish workforce retraining framework
	\end{itemize} & 
	\begin{itemize}
		\item 1,000+ veterans trained annually
		\item 15\% drop in loan defaults
		\item Engagement with 5+ state banking associations
	\end{itemize} \\
	\hline
\end{longtable}

\section{Conclusion}

Mr. Satyadhar Joshi represents precisely the type of high-impact professional that the EB2 National Interest Waiver was designed to benefit. His work enhancing the resilience of Agentic Gen AI based implementation for the U.S. financial system addresses matters of substantial merit and national importance. His unique qualifications and proven track record demonstrate that he is well-positioned to advance his proposed endeavor. Finally, the significant benefits his work provides to the United States outweigh the national interest in protecting U.S. workers through the labor certification process.

We respectfully request that USCIS approve this petition, recognizing that Mr. Joshi's contributions to financial stability, technological innovation, and workforce development provide clear and substantial benefits to the national interest of the United States.






\section{Referencing Evidences related to this Chapter }			

\textbf{Evidence Submitted}
\begin{itemize}
	\item Independent Opinion Letters from Professors, PhD and Industry experts 
	\item Evidence demonstrating the substantial merit of the endeavor
\end{itemize}

\begin{longtable}{|p{0.3\textwidth}|p{0.65\textwidth}|}
	\hline
	\textbf{RFE Concern} & \textbf{Response and Evidences } \\
	\hline
	\endhead
	Endeavor specificity & Detailed 5 year work plan with annual milestones in three different domains \\	\hline
	National importance & Linking how Mr Joshi's (applicant) research address specific issues of National Importance \\
	\hline
	Current and future impact & Downloads and Selection Citations Analysis projections \\ 	\hline
	Missing Independent  Letters & Independent Expert Opinion and Evaluations from PhDs and Professors  \\	\hline
	Impact Beyond Job at the Bank & Two Testimonial LOR of Open Access Research published in the last decade which goes beyond the job duties \\	\hline
\end{longtable}



\begin{table}[h!]
	\centering
	\renewcommand{\arraystretch}{1.4}
	\begin{tabular}{|p{6cm}|p{8cm}|}
		\hline
		\textbf{Criteria Addressed} & \textbf{Supporting Expert Letters / Testimonial Letters } \\
		\hline
		Support for Five-Year Plan & 	See Expert Opinion by Dr Asif Exhibit~\ref{chap:exhibit_asif} and 
		by Dr Anjum Exhibit~\ref{chap:exhibit_anjum}
		for details on feasibility of the proposed endeavor projections. \\
		\hline
		Evidence of Top 10\% Standing in Field & Expert Letter by Dr. Rozeria Exhibit~\ref{chap:rozeia}	and by Dr Malik Exhibit~\ref{chap:exhibit_malik}	 \\
		\hline
		Verification of DOIs, Online Profiles, and Awards & Expert Letter by Dr Sheraz Exhibit\ref{chap:sheraz} and Dr. Kamran Exhibit~\ref{chap:exhibit_kamran} \\
		\hline
		Impact beyond employer to the overall field & Testimonial Letter by Mr. Ankit Exhibit\ref{chap:exhibit_ankit} and Mr Gaurav Exhibit~\ref{chap:exhibit_gaurav} \\
		\hline
		
	\end{tabular}
	\caption{Summary of Expert Opinion Letters and Supporting Evidence.Independent  }
	\label{tab:expert_letters_summary}
\end{table}

Expert letters were obtained from professors with familiarity and experience regarding the EB2-NIW process. These experts were provided with the applicant’s EB2-NIW petition materials, including the five-year research plan and published works. Accordingly, the experts were well-positioned to evaluate the candidate’s qualifications in a comprehensive and holistic manner, and their recommendations should be considered informed and appropriate.

