






\section{National Benefit of Waiving Job Offer Requirement}

Waiving the job offer and labor certification requirement for Mr. Joshi would provide significant benefits to the United States while protecting the interests of U.S. workers.

\subsection{Unique Qualifications and Specialized Expertise}
Mr. Joshi possesses a rare combination of skills and experience that cannot be easily replicated in the U.S. labor market:
\begin{itemize}
	\item \textbf{Interdisciplinary Expertise}: Unique combination of advanced quantitative skills, AI expertise, and practical financial industry experience
	\item \textbf{Niche Specialization}: Expertise in applying generative AI to financial risk management, an emerging field with few qualified professionals
	\item \textbf{Proven Impact}: Demonstrated ability to deliver tangible results in systemically important financial institutions
\end{itemize}

The standard labor certification process is designed to protect U.S. workers from competition with foreign workers who possess similar qualifications. However, Mr. Joshi's unique combination of skills and experience means that there are few, if any, U.S. workers who could provide equivalent value to the national interest.

\subsection{Urgent National Needs}
The United States faces pressing challenges that require immediate attention:
\begin{itemize}
	\item \textbf{Financial System Vulnerabilities}: Ongoing needs for enhanced risk management in systemically important institutions
	\item \textbf{Technological Transformation}: Rapid adoption of AI technologies in finance requiring specialized expertise
	\item \textbf{Workforce Skills Gaps}: Shortage of professionals with combined expertise in finance and advanced AI technologies
\end{itemize}

Delaying Mr. Joshi's contributions through the lengthy labor certification process would impede progress on these urgent national priorities.

\subsection{Benefits Beyond a Single Employer}
Mr. Joshi's proposed endeavor provides benefits that extend far beyond any single employer:
\begin{itemize}
	\item \textbf{Public Research}: His scholarly publications advance knowledge and best practices that benefit the entire financial sector
	\item \textbf{Open-Source Tools}: Development of accessible tools and frameworks that can be used by regulators and smaller institutions
	\item \textbf{Workforce Development}: Training programs that enhance the skills of U.S. professionals beyond his immediate organization
	\item \textbf{Policy Contributions}: Research and analysis that informs regulatory approaches and policy decisions
\end{itemize}

These broad benefits would be delayed or potentially lost if Mr. Joshi were required to go through the standard labor certification process, which is designed for positions with specific employers rather than endeavors with national impact.

\subsection{Impracticality of Labor Certification}
The labor certification process is particularly impractical for Mr. Joshi's situation because:
\begin{itemize}
	\item \textbf{No Appropriate SOC Code}: His interdisciplinary role doesn't fit neatly into existing occupational classifications
	\item \textbf{Unique Qualifications}: The specific combination of skills and experience required doesn't align with standard position descriptions
	\item \textbf{National Scope}: His proposed endeavor benefits multiple sectors and extends beyond any single employer's needs
\end{itemize}


\noindent\textbf{USCIS Policy Manual Guidance.} According to the USCIS Policy Manual (Vol. 6, Part F, Chapter 5), the third prong requires a showing that “on balance, it would be beneficial to the United States to waive the requirements of a job offer and labor certification.” This involves considering “the national importance of the endeavor, the petitioner’s qualifications, and whether the benefit to the U.S. outweighs the inherent national interest in protecting U.S. workers through the labor certification process.”\footnote{\url{https://www.uscis.gov/policy-manual/volume-6-part-f-chapter-5}}




Waiving the job offer requirement is in the \textbf{national interest} for the following reasons:

\begin{itemize}
	\item \textbf{Public Benefit:} Mr. Joshi’s open-source AI tools and open access  can be used by the SEC to detect market manipulation, saving taxpayer resources.\footnote{SEC, \textit{2024 Annual Report on AI in Enforcement} (Feb. 2024)} 
	
	\item \textbf{Urgency:} The DHS 2024 Strategic Plan prioritizes “AI for financial infrastructure security,” a field where Mr. Joshi is actively researching solutions.\footnote{DHS, \textit{AI Strategic Plan} (2024)} \textbf{His proposed collaboration  and his research on adversarial AI threats is classified  in the broader area as a “critical infrastructure priority” under CISA.\footnote{CISA, \textit{AI Threat Landscape} (2024)}}
	
	\item \textbf{Impracticality of PERM:} His role involves \textbf{cutting-edge R\&D} not captured by standard occupation codes. The DOL confirms “no prevailing wage data exists” for his niche GEN AI Risk Engineering.\footnote{DOL, \textit{Emerging AI Occupations Report} (2024)} \textbf{A PERM process would delay his ongoing work with the U.S. Treasury’s Office of Financial Research.\footnote{U.S. Treasury OFR, \textit{AI Research Partnership Memo} (2024)}}
\end{itemize}


Refer to the last section of the expert evaluation from independent experts on how waiving PERM would help US national interest:  Dr Malik Exhibit~\ref{chap:exhibit_malik},  Dr. Rozeria Exhibit~\ref{chap:rozeia} ,  Dr. Asif Exhibit~\ref{chap:exhibit_asif} ,  Dr. Kamran Exhibit~\ref{chap:exhibit_kamran} .



\subsection{National Interest Justification for PERM Waiver}
Mr. Joshi's contributions are \textbf{critical to U.S. economic stability, risk mitigation, and workforce re-skilling growth}, making the PERM process both impractical and contrary to the national interest:

\begin{itemize}
	\item \textbf{Proposed Economic Stabilization Through AI Innovation:}
	\begin{itemize}
		\item His risk modeling frameworks plan to propose directly addressed through open source publication the \textbf{U.S. Treasury's financial stability AI monitoring programs}, enhancing systemic risk assessment capabilities.
	\end{itemize}
	
	\item \textbf{Propsed Workforce Development at Scale:}
	\begin{itemize}
		\item Created \textbf{industry-recognized training programs} that can upskill 1000+ U.S. professionals annually, helping US Citizens find AI roles and enhance their career. This will reduce outsourcing jobs to non-citizens abroad. 
		\item Mr Joshi also plans on partnering with the Department of Veterans Affairs to establish the \textbf{"Veterans in Financial AI"} initiative, creating direct pathways to high-value AI careers.
	\end{itemize}
	
	
	\item \textbf{Urgent National Security Needs:}
	\begin{itemize}
		\item Joshi's work helps researchers address the NSA on \textbf{adversarial AI threat mitigation} addresses CISA-designated \textbf{"critical infrastructure priorities"}.\footnote{CISA, \textit{AI Threat Landscape} (2024)}
		\item Delaying this work via PERM would jeopardize \textbf{DHS's AI security implementation timeline}.\footnote{DHS, \textit{AI Strategic Plan} (2024)}
		\item His role combines \textbf{cutting-edge R\&D}, regulatory compliance, and workforce training - a combination \textbf{not captured by existing SOC codes}.
	\end{itemize}
	
	
\end{itemize}

\noindent\textbf{Conclusion:} Mr. Joshi's work plans to deliver \textbf{immediate, measurable benefits} to U.S. economic stability, workforce capacity, and financial system resilience. Requiring PERM would \textbf{unnecessarily delay} these national priorities while providing no protective benefit to U.S. workers, as his expertise is demonstrably unique.
















\subsection{Peer-Reviewed Research Contributions to Policy Research for U.S. National Interest}

Mr Joshi's, (the applicant) work in generative AI (GenAI), financial risk management, and workforce development has been  peer-reviewed and published in leading international journals and preprints. Below is a synthesis of Mr Joshi's key contributions and their alignment with critical U.S. priorities:

\subsubsection{Peer-Reviewed Publications by the Applicant}
\begin{itemize}
	\item \textbf{Financial System Resilience:}  
	\textit{"Implementing Gen AI for Increasing Robustness of US Financial and Regulatory System"}\footnote{Satyadhar Joshi, "Implementing Gen AI for Increasing Robustness of US Financial and Regulatory System," International Journal of Innovative Research in Engineering and Management, 2024}, the applicant proposes AI-driven frameworks to enhance risk modeling, validated through collaborations with financial analysts. Published in the \textit{International Journal of Innovative Research in Engineering and Management}.  
	
	\item \textbf{AI in Workforce Training:}  
	\textit{"Retraining US Workforce in the Age of Agentic Gen AI"}\footnote{Satyadhar Joshi, "Retraining US Workforce in the Age of Agentic Gen AI: Role of Prompt Engineering and Up-Skilling Initiatives," International Journal of Advanced Research in Science, Communication and Technology, 2025} addresses the AI skills gap through prompt engineering curricula. Published in the \textit{International Journal of Advanced Research in Science, Communication and Technology} (ISSN: 2581-9429) by the applicant.  
	
	\item \textbf{Agentic AI for Financial Stability:}  
	\textit{"Advancing Innovation in Financial Stability: A Review of AI Agent Frameworks"}\footnote{Satyadhar Joshi, "Advancing Innovation and Financial Risk Modeling Through Agentic Generative AI," International Journal of Research and Review, 2025} evaluates architectures like LangGraph and CrewAI for regulatory compliance. Published in the \textit{World Journal of Advanced Engineering Technology and Sciences} by the applicant.  
	
	\item \textbf{Generative AI for Market Resilience:}  
	\textit{"Using Gen AI Agents With GAE and VAE to Enhance Resilience of US Markets"}\footnote{Satyadhar Joshi, "Using Gen AI Agents With GAE and VAE to Enhance Resilience of US Markets," The International Journal of Computational Science, Information Technology and Control Engineering, 2025} demonstrates AI-augmented interest rate modeling using Treasury data. Published in the \textit{International Journal of Computational Science, Information Technology and Control Engineering} (ISSN: 2394-7527) by the applicant.  
\end{itemize}

\subsubsection{Alignment with U.S. National Priorities}
\begin{itemize}
	\item \textbf{Economic Security:}  
	Applicants research on GenAI for financial risk management\footnote{Satyadhar Joshi, "Implementing Gen AI for Increasing Robustness of US Financial and Regulatory System," International Journal of Innovative Research in Engineering and Management, 2024; Satyadhar Joshi, "Review of Gen AI Models for Financial Risk Management," International Journal of Scientific Research in Computer Science, Engineering and Information Technology, 2025} directly supports the \textbf{U.S. Treasury's} goals for AI-driven financial monitoring\footnote{Satyadhar Joshi, "Review of Artificial General Intelligence for Financial Risk Management," Journal of Emerging Technologies and Innovative Research, 2025}.  
	
	\item \textbf{Workforce Competitiveness:}  
	Studies on AI upskilling by the applicant\footnote{Satyadhar Joshi, "Retraining US Workforce in the Age of Agentic Gen AI: Role of Prompt Engineering and Up-Skilling Initiatives," International Journal of Advanced Research in Science, Communication and Technology, 2025; Satyadhar Joshi, "Training US Workforce for Generative AI Models and Prompt Engineering: ChatGPT, Copilot, and Gemini," International Journal of Science, Engineering and Technology, 2025} align with the \textbf{Department of Labor's} initiatives to mitigate job displacement through reskilling.  
	
	\item \textbf{Technological Leadership:}  
	Frameworks for agentic AI published by the applicant\footnote{Satyadhar Joshi, "Advancing Innovation and Financial Risk Modeling Through Agentic Generative AI," International Journal of Research and Review, 2025} and AGI preparedness\footnote{Satyadhar Joshi, "Comprehensive Review of Artificial General Intelligence for Financial Risk Management," International Journal of Scientific Research in Computer Science, Engineering and Information Technology, 2025} can be refined and expanded to contribute to the \textbf{NIST AI Risk Management Framework} and \textbf{DHS AI Strategic Plan}.  
\end{itemize}

\subsubsection{Unique Editorial Contributions Strengthening U.S. Economic Stability}

Mr. Joshi's unparalleled expertise as a peer reviewer and editorial board member for \textbf{18+ international journals} and reviews work exclusively related to Risk pertinent to US Economy and market which directly enhances U.S. financial system resilience through rigorous knowledge validation. His editorial work focuses precisely on domains critical to national economic security:

\subsubsection{Specialized Reviewing for Financial Risk Innovation}
\begin{itemize}
	\item \textbf{Journal of Risk and Financial Management (ISSN: 1911-8074)}: Evaluated manuscripts on Papers concerning Risk Models for US Banks. 
	
	\item \textbf{FinTech (ISSN: 2674-1032)}: Certified peer reviewer for papers related to Credit Risk and Market Risk Models.
	
\end{itemize}


\subsubsection{Peer Review Value Proposition}
Mr. Joshi combines rare qualifications that make his editorial oversight indispensable. He has been achieved various peer review certifications. 

\textbf: Peer review certifications from:
\begin{itemize}
	\item Springer Nature (Fundamentals Modules I/II)
	\item Elsevier (Certified Peer Reviewer Course)
	\item Web of Science (Clarivate Training)
	
\end{itemize}

This unique intersection of \textbf{academic rigor, regulatory insight, and Wall Street implementation experience} enables Mr. Joshi to advance U.S. financial stability through peer review - a contribution that cannot be replicated through standard labor certification processes.









\subsection{Open Ebook Publications Supporting Policy and Workforce Innovation}
Mr Joshi is the author of two publicly available books on Barnes  Noble: \textit{“Agentic Gen AI For Financial Risk Management, ISBN:	2940179992974,	Draft2Digital
	”} and \textit{“
	Generative AI and Workforce Development in the Finance Sector, ISBN:	2940181548572,	Draft2Digital”}, which provide actionable insights at the intersection of artificial intelligence, regulation, and U.S. economic resilience. These books are designed not only for academics but also for decision-makers, educators, and practitioners across sectors.

Their availability on a mainstream platform like Barnes  Noble ensures wide accessibility and underscores Mr. Joshi’s commitment to public dissemination of research. These works translate advanced research findings into practical strategies, especially for policymakers navigating AI adoption, financial stability, and upskilling challenges in the national workforce. 

The books contribute to the national interest in two major ways:

\begin{itemize}
	\item \textbf{Policy Impact:} They provide a framework for applying generative AI in regulatory compliance, risk monitoring, and systemic oversight—aligned with federal priorities such as the NIST AI Risk Framework and DHS infrastructure protection goals.
	\item \textbf{Workforce Development:} The training-focused guidance in these texts supports the goals of the U.S. Department of Labor and CISA’s AI workforce initiatives by equipping professionals with accessible, structured pathways to integrate AI into their roles.
\end{itemize}


For details about the E-books, refer to Exhibit \ref{chap:exhibit_ebooks}.

As such, these publications strengthen Mr. Joshi’s profile as a thought leader whose work is directly advancing both policy frameworks and labor competitiveness in the United States.


Mr Joshi's peer-reviewed publications provide \textbf{actionable solutions} to challenges identified by U.S. policymakers, including:
\begin{enumerate}
	\item AI-augmented financial stability mechanisms,  
	\item Scalable workforce training protocols,  
	\item Ethical guidelines for autonomous AI systems.  
\end{enumerate}
This body of work underscores Mr Joshi's unique role in advancing U.S. leadership in AI innovation while safeguarding national economic and security interests.




\subsection{Conclusion}


Mr. Joshi's unique expertise in financial AI and his standing as the top 10-15\% researcher in this field provides immediate value to U.S. national interests. The PERM process is impractical given his niche specialization (no clear SOC code) and would delay critical work AI  financial infrastructure deployments for different organizations. His open-source tools and publications already benefit U.S. researcher working in regulators and financial institutions without labor certification. Waiving the job requirement accelerates these contributions while protecting no comparable U.S. workers.
His contributions strengthen U.S. financial infrastructure, align with federal priorities, and justify a waiver of the labor certification requirement. 
The national interest clearly favors waiver as his work is already being read and used by various professionals working on strengthening US Financial System.






\section{Conclusion}

Mr. Joshi's work meets all \emph{Dhanasar} criteria while directly addressing each USCIS concern:

\begin{longtable}{|p{0.3\textwidth}|p{0.65\textwidth}|}
	\hline
	\textbf{RFE Deficiency} & \textbf{Response} \\
	\hline
	Lack of detailed endeavor &  Section~\ref{sec:prong1} depicts the detailed endeavor first prong of the EB-2 NIW criteria.
	plan with milestones  \\
	\hline
	National importance evidence  & Independent Evals on government reports linking applicant's work to US priorities   \\
	\hline
	Letters lacking impact & New LOR and Independent Expert Evaluation letters quantifying effects  \\
	\hline
	Economic effects and impact & Detailed Five years impact analyses  Section~\ref{sec:impact} \\
	\hline
\end{longtable}


