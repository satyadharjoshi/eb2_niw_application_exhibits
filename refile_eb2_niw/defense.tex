\section{National Importance: Critical Contributions to U.S. Defense and Cybersecurity}
\label{sec:defense}

Mr. Joshi's expertise extends beyond financial stability into the critical national security domains of defense and cybersecurity. His pioneering research in Secure Multi-Party Computation (SMC), Agentic AI, and High-Performance Computing (HPC) provides tangible, advanced solutions for U.S. defense applications, making his continued work in the United States a matter of strategic importance.

\subsection{Securing Military Operations with Advanced Cryptography}

Mr. Joshi's foundational work in Secure Multi-Party Computation (SMC) addresses a core challenge in modern military and intelligence operations: enabling secure collaboration between multiple parties without exposing sensitive underlying data. His research, specifically cited in defense literature, has direct applications for the U.S. Department of Defense (DoD) and intelligence communities.

\begin{itemize}
	\item \textbf{Protocol for Defense Applications:} In his peer-reviewed publication, \textit{``Secure Multi-party Computation Protocol for Defense Applications in Military Operations Using Virtual Cryptography,''} Mr. Joshi developed a protocol that allows separate military units or allied nations to jointly compute a strategy or analyze intelligence without sharing their proprietary or classified data sets.\footnote{R. Pathak and S. Joshi, ``Secure Multi-party Computation Protocol for Defense Applications in Military Operations Using Virtual Cryptography,'' in \textit{Contemporary Computing}, S. Ranka et al., Eds., Berlin, Heidelberg: Springer, 2009, pp. 389–399. doi: 10.1007/978-3-642-03547-0\_37.}
	\item \textbf{Strategic Relevance:} This technology is critical for modern joint-allied operations, secure logistics planning, and multi-agency threat analysis, directly supporting the DoD's imperative for secure information sharing outlined in its Cyber Strategy. His work provides a technical foundation for maintaining operational security while enhancing collaborative effectiveness.
\end{itemize}

\subsection{Advancing Cybersecurity through Agentic AI and HPC}

The U.S. faces persistent and evolving cyber threats from state and non-state actors targeting critical infrastructure and defense networks. Mr. Joshi's recent research focuses on leveraging the combined power of Agentic AI and High-Performance Computing to proactively defend these systems.

\begin{itemize}
	\item \textbf{Proactive Cyber Defense:} In his 2025 work, \textit{``Advancing Cybersecurity Through Synergies of Agentic AI and High-Performance Computing,''} Mr. Joshi explores how autonomous AI agents, powered by HPC, can predict, detect, and respond to sophisticated cyber-attacks at a speed and scale unattainable by human operators alone.\footnote{S. Joshi, ``Advancing Cybersecurity Through Synergies of Agentic AI and High-Performance Computing,'' vol. 02, no. 07, 2025.}
	\item \textbf{Comprehensive Review of Architectures:} His publication, \textit{``Gen AI in Financial Cybersecurity: A Comprehensive Review of Architectures, Algorithms, and Regulatory Challenges,''} while focused on finance, provides a framework that is directly transferable to securing defense critical infrastructure (DCI).\footnote{S. Joshi, ``Gen AI in Financial Cybersecurity: A Comprehensive Review of Architectures, Algorithms, and Regulatory Challenges,'' \textit{International Journal of Innovations in Science, Engineering And Management}, pp. 73–88, July 2025. doi: 10.69968/ijisem.2025v4i373-88.} The architectures and algorithms analyzed are essential for protecting military command and control systems, weapons platforms, and sensitive research data from advanced persistent threats (APTs).
\end{itemize}

\subsection{Strategic Analysis of Defense Partnerships}

Mr. Joshi's analytical skills are further demonstrated by his quantitative and qualitative analysis of international defense relationships. His co-authored study, \textit{``India – Israel Defense Relationship,''} involved a detailed assessment of defense companies and collaboration patterns.\footnote{A. Ludhiyani and S. Joshi, ``India – Israel Defense Relationship: Quantitative \& Qualitative Analysis of Defense Companies of India and Israel,'' \textit{Journal of Defense Studies and Resource Management}, vol. 2015, May 2016. doi: 10.4172/2324-9315.1000120.}\footnote{A. Ludhiyani, S. Joshi, R. Pathak, P. Parandkar, and S. Katiyal, ``Subjective and assessable exploration of India-Israel defense relationship,'' in \textit{2015 2nd International Conference on Computing for Sustainable Global Development (INDIACom)}, Mar. 2015, pp. 314–319. doi: 10.1109/INDIACom.2015.7100265.} This showcases his ability to conduct strategic-level analysis that informs understanding of global defense markets and technological transfer—a skillset of immense value to U.S. defense policy and intelligence agencies.

\subsection{Alignment with U.S. Defense Priorities}

Mr. Joshi's proposed endeavor to establish a policy research center would allow him to directly apply this unique blend of technical and strategic expertise to pressing U.S. national security needs:

\begin{itemize}
	\item \textbf{Developing Next-Generation Tools:} He can advance his SMC and Agentic AI research to create open-source or government-purpose tools for the DoD and DHS.
	\item \textbf{Informing Policy:} His center can produce unclassified policy reports on AI and cybersecurity threats, providing actionable insights for legislators and agency leaders.
	\item \textbf{Upskilling the Defense Workforce:} His proven training methodologies can be adapted to create certification programs for military personnel and defense contractors in secure AI deployment and cyber defense, directly supporting the U.S. government's initiative to build a cyber-ready workforce.\footnote{Satyadhar Joshi, ``The Impact of AI on Veteran Employment and the Future Workforce Development: Opportunities, Barriers, and Systemic Solutions,'' \textit{World J. Adv. Res. Rev.}, vol. 27, no. 2, pp. 328–341, Sept. 2025. doi: 10.30574/wjarr.2025.27.3.3147.}
\end{itemize}

\subsection{Conclusion on Defense Criticality}

Mr. Joshi is not merely a researcher in a relevant field; he is a proven innovator with published, applicable research in areas that the U.S. government identifies as critical to maintaining its technological and strategic advantage. His work on secure computation and AI-driven cybersecurity provides concrete solutions to real-world defense challenges. Allowing him to continue and expand this work freely in the United States, unconstrained by a specific employer's focus, is unequivocally in the national interest of the United States. His contributions are poised to strengthen national security, enhance military capabilities, and protect critical infrastructure from emerging threats.