



\label{subsec:ppi}

The \textit{Matter of Dhanasar} framework requires demonstrating that a proposed endeavor has "potential prospective impact" (PPI) with "substantial positive effects" for the United States. The USCIS denial questioned whether Mr. Joshi's work would have implications beyond his immediate role or clientele. This chapter provides comprehensive evidence that Mr. Joshi's endeavor has precisely the type of broad, quantifiable, and nationally important impact contemplated by \textit{Dhanasar}.

\section{Quantified Impact Projections}

Based on current adoption rates, historical growth trajectories, and validation from independent experts, Mr. Joshi's endeavor demonstrates substantial prospective impact across multiple dimensions:

\subsection{Proposed Possible Economic Impact}
\begin{itemize}
	\item \textbf{Compliance Cost Reduction:} AI-driven regulatory automation projects 15–25\% reduction in compliance costs for mid-size banks, translating to \$2–5 million annually per institution
	\item \textbf{Systemic Risk Mitigation:} Enhanced risk modeling frameworks could prevent losses similar to the 2008 crisis, where inadequate risk modeling contributed to \$2.8 trillion in economic damage
	\item \textbf{Productivity Gains:} 30–40\% productivity improvements in financial services operations through AI automation and process optimization
	\item \textbf{Capital Efficiency:} 25–40\% reduction in capital allocation inefficiencies across U.S. banking sector, potentially freeing billions for productive lending
\end{itemize}

\subsection{Proposed Possible Workforce Impact}
\begin{itemize}
	\item \textbf{Professional Training:} 5,000+ U.S. professionals trained by 2030 through structured programs and workshops
	\item \textbf{Veteran Focus:} "Veterans in Financial AI" program targeting 500+ veterans annually, with 85\% placement rate in fintech roles
	\item \textbf{Salary Impact:} Trained participants earning average of \$15,000 more annually in AI-enhanced financial roles
	\item \textbf{Geographic Reach:} Establishment of 2–5 regional training hubs to ensure nationwide impact
\end{itemize}

\subsection{Proposed Possible Regulatory and Policy Impact}
\begin{itemize}
	\item \textbf{Tool Adoption:} Open-source FinRisk-AI toolkit already downloaded 5,000+ times, with projected adoption by 50+ financial institutions
	\item \textbf{Agency Engagement:} Formal collaborations with FDIC, OCC, and Federal Reserve on AI implementation frameworks
	\item \textbf{Policy Contributions:} 2–3 commentary letters annually submitted to SEC, CFPB, and FSOC on AI regulation
	\item \textbf{Standards Development:} Contributions to NIST AI Risk Management Framework and industry standards (IEEE, ISO)
\end{itemize}

\subsection{Proposed Possible Research and Knowledge Dissemination}
\begin{itemize}
	\item \textbf{Publication Reach:} 75,000+ cumulative downloads of publications by 2030, from current baseline of 20,000+ downloads
	\item \textbf{Academic Integration:} Research integration into curricula at 10+ U.S. universities and community colleges
	\item \textbf{Global Recognition:} Work featured in international policy outlets (Impacto TIC, LLRX.com) reaching hundreds of thousands of readers
	\item \textbf{Citation Growth:} Projected increase from current 800+ citations to 2,000+ by 2030 based on current trajectory
\end{itemize}

\section{Validation by Independent Experts}

The reasonableness of these projections is confirmed by multiple independent experts :

\subsection{Dr. Mohd Anjum}
\begin{quote}
	\textit{"Mr. Joshi's detailed roadmap demonstrates a clear capacity to execute his proposed research and training initiatives, which are of significant merit and national importance. His projections of training 5,000+ professionals and achieving 75,000+ publication downloads are conservative estimates based on his current trajectory of impact."}  
\end{quote}

\subsection{Dr. Malik Missan }
\begin{quote}
	\textit{"His projections are supported by his existing publication record, peer review contributions, and growing influence in the field—indicators of his ability to advance the endeavor as planned. The economic impact projections of \$2-5M savings per institution are realistic given the demonstrated efficiency gains from AI automation in financial compliance."}  
\end{quote}

\subsection{Dr. Asif Umer }
\begin{quote}
	\textit{"His proposed five-year endeavor is both realistic and highly impactful, aligning with U.S. financial and technological priorities. The workforce development targets are particularly achievable given the documented shortage of AI talent in financial services and Mr. Joshi's proven ability to create effective training programs."}
\end{quote}

\section{Evidence-Based Projection Methodology}

The impact projections are not speculative but based on:

\subsection{Historical Growth Trends}
\begin{itemize}
	\item \textbf{Research Downloads:} Current annual rate of 15,000–20,000 downloads, growing at 25\% annually
	\item \textbf{Training Participation:} Current programs reaching 1,000+ annually, with 40\% year-over-year growth
	\item \textbf{Tool Adoption:} Open-source toolkit downloaded 5,000+ times in first year, with accelerating adoption
\end{itemize}

\subsection{Market Demand Indicators}
\begin{itemize}
	\item \textbf{BLS Data:} 50\% growth in AI-finance job postings with insufficient qualified candidates
	\item \textbf{Industry Surveys:} 78\% of financial institutions reporting AI skills gaps affecting operations
	\item \textbf{Regulatory Mandates:} FSOC 2023 Annual Report highlighting AI as "transformative tool for systemic risk monitoring"
\end{itemize}

\subsection{Government Priority Alignment}
\begin{itemize}
	\item \textbf{White House Initiatives:} America's AI Action Plan (July 2025) outlining 90+ federal policy actions
	\item \textbf{Treasury Priorities:} 2025 AI Report emphasizing need for AI modernization in financial services
	\item \textbf{DHS Framework:} Roles and Responsibilities Framework for AI in Critical Infrastructure (2025)
\end{itemize}

\section{Comparative Impact Assessment}

To contextualize the national importance of Mr. Joshi's projected impact:

\begin{longtable}{|p{0.3\textwidth}|p{0.3\textwidth}|p{0.3\textwidth}|}
	\hline
	\textbf{Impact Category} & \textbf{Mr. Joshi's Projected Impact} & \textbf{National Significance} \\
	\hline
	\textbf{Financial Stability} & 15–25\% improved risk model accuracy & Addresses FSOC priority on systemic risk monitoring \\
	\hline
	\textbf{Workforce Development} & 5,000+ professionals trained & Supports DOL goal of closing AI skills gap in finance \\
	\hline
	\textbf{Regulatory Efficiency} & 30–50\% faster compliance processes & Aligns with Treasury focus on regulatory modernization \\
	\hline
	\textbf{Economic Savings} & \$2–5M per institution annually & Contributes to national economic competitiveness \\
	\hline
	\textbf{Research Contribution} & 75,000+ publication downloads & Advances U.S. leadership in financial AI research \\
	\hline
\end{longtable}

\section{Risk Mitigation and Contingency Planning}

The projected impact accounts for potential implementation challenges:

\subsection{Funding Variability}
\begin{itemize}
	\item \textbf{Mitigation:} Diversified funding sources including grants, industry partnerships, and university support
	\item \textbf{Contingency:} Scalable program design allowing for adjustment based on available resources
\end{itemize}

\subsection{Regulatory Changes}
\begin{itemize}
	\item \textbf{Mitigation:} Focus on foundational AI risk principles relevant across regulatory regimes
	\item \item \textbf{Contingency:} Modular framework design allowing rapid adaptation to new requirements
\end{itemize}

\subsection{Technology Evolution}
\begin{itemize}
	\item \textbf{Mitigation:} Open-source, modular tools that can be updated as AI technology advances
	\item \textbf{Contingency:} Ongoing research commitment ensuring methodologies remain state-of-the-art
\end{itemize}

\section{International Govt / Quasi Govt Citation of Work}

Mr. Joshi's research has been cited in an article published through \textbf{SciEngine} (China Science Publishing \& Media Ltd.), a state-owned academic publishing platform in China. SciEngine is operated by China Science Publishing \& Media Ltd. (Science Press), one of the largest and most reputable academic publishers in China, which is affiliated with the Chinese Academy of Sciences.

Being cited in a SciEngine-published journal underscores the \textbf{international recognition and influence} of Mr. Joshi's research, since SciEngine hosts peer-reviewed journals that are widely disseminated and indexed. This citation demonstrates that his work has been relied upon by other scholars, including in government-affiliated publishing outlets, thereby strengthening the evidence of its importance in the field.

For reference, the article citing Mr. Joshi's work is available at: \url{https://www.sciengine.com/BNSFC/doi/10.3724/BNSFC-2025.04.20.0001}.


\section*{Mr Joshi's Publication on CORE UK}

Mr. S. Joshi's article, \textit{``Review of Artificial Intelligence in Management, Leadership, and Decision-Making''} (2025), is hosted on \textbf{CORE (UK)}, the United Kingdom's open-access research aggregator. CORE.ac.uk is operated by the Knowledge Media Institute at The Open University and is widely used by UK research bodies, including UK Research and Innovation (UKRI) and Research England, for accessing and monitoring scholarly publications.

Accessibility through \textbf{CORE (UK)} broadens the reach of Mr. Joshi's work to an international audience of academics, policymakers, and industry stakeholders. Its inclusion on this platform demonstrates the visibility and potential dissemination impact of his research on AI applications in management and decision-making.

\section*{Mr Joshi's Publication on Ukrainian Government Research Platform}

Mr. S. Joshi's article, \textit{``Artificial Intelligence in Conflict Resolution: A Comprehensive Review of Techniques and Applications''} (2025), is hosted on the \textbf{Open Ukrainian Citation Index (OUCI)}. OUCI is operated by the \textbf{State Scientific and Technical Library of Ukraine (DNTB)}, under the authority of the Ukrainian government, and provides open access to national and international scholarly publications.

Indexing on this official Ukrainian platform increases the accessibility and visibility of Mr. Joshi's research, highlighting its relevance to global discussions on artificial intelligence and conflict resolution.

Exhibit government citations ~\ref{chap:gov}


\section{Conclusion: Compelling Evidence of National Impact}

The potential prospective impact of Mr. Joshi's endeavor is substantial, quantifiable, and directly aligned with national priorities. The evidence demonstrates:

\begin{enumerate}
	\item \textbf{Quantifiable Projections:} Specific, measurable impact targets across economic, workforce, regulatory, and research domains
	\item \textbf{Expert Validation:} Independent confirmation of reasonableness by multiple domain experts
	\item \textbf{Historical Basis:} Projections grounded in current performance and growth trajectories
	\item \textbf{Market Alignment:} Responsiveness to documented needs and demands in the financial sector
	\item \textbf{Government Priority:} Direct support for multiple federal initiatives and policy goals
	\item \textbf{Risk Management:} Thoughtful consideration of potential challenges and mitigation strategies
\end{enumerate}

This comprehensive evidence establishes that Mr. Joshi's endeavor has the "potential prospective impact" required by \textit{Dhanasar}, with "substantial positive effects" that will benefit the United States through enhanced financial stability, workforce development, regulatory efficiency, and economic competitiveness. The waiver of the job offer requirement is essential to maximize this nationally important impact.
